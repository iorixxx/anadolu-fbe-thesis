%   Copyright 2016 Ahmet Arslan
%
%   Licensed under the Apache License, Version 2.0 (the "License");
%   you may not use this file except in compliance with the License.
%   You may obtain a copy of the License at
%
%       http://www.apache.org/licenses/LICENSE-2.0
%
%   Unless required by applicable law or agreed to in writing, software
%   distributed under the License is distributed on an "AS IS" BASIS,
%   WITHOUT WARRANTIES OR CONDITIONS OF ANY KIND, either express or implied.
%   See the License for the specific language governing permissions and
%   limitations under the License.

\newpage
\begin{spacing}{1.1}
\addcontentsline{toc}{chapter}{\"{O}ZET}
\begin{center}
\textbf{\"{O}ZET} \vspace{4mm}\\
BELGE DERLEMLER\.{I}NDE SORGU TER\.{I}MLER\.{I}N\.{I}N FREKANS DA\u{G}ILIMLARININ ANAL\.{I}Z\.{I} VE SORGUYA G\"{O}RE EN UYGUN TER\.{I}M A\u{G}IRLIKLANDIRMA MODEL\.{I}N\.{I}N SE\c{C}\.{I}M\.{I}\\
\vspace{4mm}
Ahmet ARSLAN
\vspace{4mm} \\
Bilgisayar M\"{u}hendisli\u{g}i Anabilim Dal{\i} \\
Anadolu \"{U}niversitesi, Fen Bilimleri Enstit\"{u}s\"{u}, A\u{g}ustos, 2016 \\
\vspace{4mm}
Dan{\i}\c{s}man: Do\c{c}. Dr. Bekir Taner D\.{I}N\c{C}ER \\
\end{center}
\vspace*{-1mm}


Bilgi eri\c{s}imi i\c{c}in bir \c{c}ok terim a\u{g}{\i}rl{\i}kland{\i}rma modeli geli\c{s}tirilmi\c{s}tir. 
Fakat her terim a\u{g}{\i}rl{\i}kland{\i}rma modelinin ba\c{s}ar{\i}m{\i} baz{\i} sorgularda y\"{u}ksek baz{\i} sorgularda da d\"{u}\c{s}\"{u}kt\"{u}r --- ba\c{s}ar{\i}m{\i}n g\"{u}rb\"{u}zl\"{u}\u{g}\"{u} problemi. 
Di\u{g}er taraftan bir terim a\u{g}{\i}rl{\i}kland{\i}rma modelinin ba\c{s}ar{\i}m{\i}n{\i}n d\"{u}\c{s}\"{u}k oldu\u{g}u bir sorgu i\c{c}in di\u{g}er terim a\u{g}{\i}rl{\i}k-land{\i}rma modellerinin ba\c{s}ar{\i}m{\i} da d\"{u}\c{s}\"{u}k olmak zorunda de\u{g}ildir: herhangi bir sorgu i\c{c}in tatminkar d\"{u}zeyde ba\c{s}ar{\i}m sa\u{g}layacak bir terim a\u{g}{\i}rl{\i}kland{\i}rma modelini mevcut teknolojiler i\c{c}inde bulmak m\"{u}mk\"{u}n olabilir. Yani sisteme gelen her sorguyu tek bir terim a\u{g}{\i}rl{\i}kland{\i}rma modeli ile cevaplamak, kullan{\i}c{\i}lar{\i}n bilgi ihtiya\c{c}lar{\i}n{\i} en tatminkar \c{s}ekilde kar\c{s}{\i}lamak i\c{c}in uygun olmayabilir. T\"{u}m sorgular i\c{c}in tekil bir terim a\u{g}{\i}rl{\i}kland{\i}rma modeli kullanmak yerine, her bir ayr{\i} sorgu i\c{c}in uygun bir terim a\u{g}{\i}rl{\i}kland{\i}rma modeli kullan{\i}ld{\i}\u{g}{\i}nda bilgi eri\c{s}im ba\c{s}ar{\i}m{\i}n{\i}n mertebe kertesinde art{\i}\c{s} oldu\u{g}u deneysel bir ger\c{c}ektir. Ancak, verilen herhangi bir sorgu i\c{c}in en iyi ba\c{s}ar{\i}m{\i} sa\u{g}layacak olan modelin, bug\"{u}nk\"{u} bilinen en geli\c{s}kin modeller aras{\i}ndan otomatik olarak se\c{c}iminin yap{\i}lmas{\i} i\c{s}i halen \c{c}\"{o}z\"{u}lememi\c{s} zor bir ara\c{s}t{\i}rma konusudur. Bu u\u{g}ra\c{s}, se\c{c}kili bilgi eri\c{s}imi \c{c}al{\i}\c{s}ma alan{\i}nda, genel olarak, se\c{c}kili terim a\u{g}{\i}rl{\i}kland{\i}rma ya da se\c{c}kili a\u{g}{\i}rl{\i}kland{\i}rma fonksiyonu olarak adland{\i}r{\i}l{\i}r. Bu doktora tezinde, se\c{c}kili terim a\u{g}{\i}rl{\i}kland{\i}rma u\u{g}ra\c{s}{\i} i\c{c}in sorgu terimlerinin derlemler \"{u}zerindeki frekans da\u{g}{\i}l{\i}mlar{\i}na dayanan \"{o}zg\"{u}n bir istatistik/olas{\i}l{\i}k esas{\i}nda yakla\c{s}{\i}m incelenmi\c{s}tir.

Bir sorguda iyi \c{c}al{\i}\c{s}an terim a\u{g}{\i}rl{\i}kland{\i}rma modeli ba\c{s}ka bir sorguda iyi \c{c}al{\i}\c{s}mayabilmektedir. 
Verilen herhangi bir sorgunun en iyi \c{c}al{\i}\c{s}aca\u{g}{\i} terim a\u{g}{\i}rl{\i}k-land{\i}rma modelini \"{o}nceden belirleyemiyoruz. 
Terim a\u{g}{\i}rl{\i}kland{\i}rma modellerinin ba\c{s}ar{\i}m{\i} \"{u}zerine etki eden sorgu ve derlem karakteristikleri hakk{\i}nda \c{c}ok az bilgiye sahibiz. 
Bu doktora tezinde, s\"{o}z konusu gizeme bir nebze olsun {\i}\c{s}{\i}k tutmak ama\c{c}lanmaktad{\i}r.

Bu tezde sunulan b\"{u}t\"{u}n deney sonu\c{c}lar{\i}n{\i} tekrarlamak ve yeniden \"{u}retmek i\c{c}in gerekli olan veri ve kod \c{c}evrimi\c{c}i olarak mevcuttur.\

\setlength\leftmargini{4.0cm}
\begin{enumerate}[label=\textbf{Anahtar S\"{o}zc\"{u}kler:}]
  \item Ki-Kare Testi, \.{I}ndeks Terim A\u{g}{\i}rl{\i}kland{\i}rma, Frekans Da\u{g}{\i}l{\i}m{\i}, Bilgi Eri\c{s}imde Ba\c{s}ar{\i}m G\"{u}rb\"{u}zl\"{u}\u{g}\"{u} Problemi, Se\c{c}kili Bilgi Eri\c{s}im.
\end{enumerate}

\end{spacing}
