% Chapter 9 - Conclusions
\documentclass[a4paper,12pt,oneside,authoryear]{report}
\usepackage{natbib}
\usepackage[latin5]{inputenc}
\usepackage{amsmath}
\usepackage{MnSymbol}
\usepackage{wasysym}
\usepackage{algpseudocode}
\usepackage{multirow}
\usepackage[font={small},format=hang,labelfont=bf,justification=justified,singlelinecheck=false]{caption}
\usepackage{epsfig}
\usepackage[subfigure,titles]{tocloft}
\usepackage[tight,footnotesize]{subfigure}
\usepackage{array}
\usepackage{epigraph}
\usepackage{url}
\usepackage{fixltx2e}
\usepackage{siunitx}
\usepackage{listings}
\usepackage{csquotes}
\usepackage{lscape}
\usepackage{algorithm2e}
\newcommand*\pct{\scalebox{.9}{\%}}
\usepackage[maxfloats=25]{morefloats}
\usepackage[shortlabels]{enumitem}
% margins - 3.5cm from left, 2.5 cm from right, 3cm from top, and 2.5cm from bottom 
\usepackage[top=30mm, bottom=25mm, left=35mm, right=25mm]{geometry}
\usepackage{hyperref}
\usepackage[labelsep=period]{caption}


\usepackage{color}
\definecolor{gray}{rgb}{0.4,0.4,0.4}
\definecolor{darkblue}{rgb}{0.0,0.0,0.6}
\definecolor{cyan}{rgb}{0.0,0.6,0.6}
\definecolor{lightpurple}{rgb}{0.8,0.8,1}

\newcommand*\ac{[}
\newcommand*\kapa{]}


\lstdefinelanguage{XML}
{
  morestring=[b]",
    showstringspaces=false,
  morestring=[s]{>}{<},
  morecomment=[s]{<?}{?>},
  stringstyle=\color{black},
  identifierstyle=\color{cyan},
  keywordstyle=\color{darkblue},
  morekeywords={topic,query,description}% list your attributes here
}

% set line spacing to 1.5
\usepackage{setspace}
\onehalfspacing


\pagestyle{plain}
\setlength{\parskip}{2mm}
\setlength{\itemsep}{1mm}
\setcounter{page}{1}
\parindent 10mm

\usepackage[english]{babel}
\usepackage{titlesec}
\newcommand{\hsp}{\hspace{0pt}}
\titleformat{\chapter}[hang]{\large\bfseries}{\thechapter\hsp{.}\hsp}{10pt}{\large\bfseries}
\titlespacing*{\chapter}{0pt}{0pt}{0pt}
\titleformat{\section}[hang]{\large\bfseries}{\thesection\hsp{.}\hsp}{10pt}{\large\bfseries}
\titlespacing*{\section}{0pt}{0pt}{0pt}
\titleformat{\subsection}[hang]{\normalsize\bfseries}{\thesubsection\hsp{.}\hsp}{10pt}{\normalsize\bfseries}
\titlespacing*{\subsection}{0pt}{0pt}{0pt}



\setcounter{tocdepth}{3}
\setcounter{secnumdepth}{3}
\cftsetindents{section}{1.6em}{2.8em}
\cftsetindents{subsection}{4.4em}{3.8em}


\renewcommand{\cftfigfont}{\textbf{Figure} }
\renewcommand{\cfttabfont}{\textbf{Table} }

\renewcommand{\cftfigaftersnum}{.}
\renewcommand{\cfttabaftersnum}{.}


\renewcommand{\cftchapleader}{\cftdotfill{\cftdotsep}} % for chapters


\renewcommand{\cftchapaftersnum}{.}
\renewcommand{\cftsecaftersnum}{.}
\renewcommand{\cftsubsecaftersnum}{.}

\renewcommand\cftchappagefont{\bfseries}
\renewcommand\cftsecpagefont{\bfseries}
\renewcommand\cftsubsecpagefont{\bfseries}

\renewcommand{\cftchapleader}{\bfseries\cftdotfill{\cftsecdotsep}}% dot leaders in bold
\renewcommand{\cftsecleader}{\bfseries\cftdotfill{\cftsecdotsep}}% dot leaders in bold
\renewcommand{\cftsubsecleader}{\bfseries\cftdotfill{\cftsecdotsep}}% dot leaders in bold


\cftsetindents{figure}{0em}{3.0em}
\cftsetindents{table}{0em}{3.0em}

\AtBeginDocument{%
\addtocontents{toc}{~\hfill\textbf{\underline{Page}}\par}
\addtocontents{lof}{~\hfill\textbf{\underline{Page}}\par}
\addtocontents{lot}{~\hfill\textbf{\underline{Page}}\par}
}

\makeatletter
\g@addto@macro\cfttabpresnum{\bfseries} % make figure number bold in 'list of figures'
\g@addto@macro\cftfigpresnum{\bfseries}  % make table number bold in 'list of table'

\g@addto@macro\cftsecfont{\bfseries}
\g@addto@macro\cftsubsecfont{\bfseries}
\g@addto@macro\cftparafont{\bfseries}
\g@addto@macro\cftsubparafont{\bfseries}
%\g@addto@macro\cftfigfont{\bfseries}
%\g@addto@macro\cfttabfont{\bfseries}

\g@addto@macro\cftsecpagefont{\bfseries}
\g@addto@macro\cftsubsecpagefont{\bfseries}
\g@addto@macro\cftparapagefont{\bfseries} 
\g@addto@macro\cftsubparapagefont{\bfseries}
%\g@addto@macro\cftfigpagefont{\bfseries}
%\g@addto@macro\cfttabpagefont{\bfseries}
\makeatother
\begin{document}

\chapter{\textbf{CONCLUDING REMARKS}}
\label{ch9}

\section{Contributions and Conclusions}

There has been a great deal of research dedicated to develop term-weighting models for information retrieval (IR). 
However, IR research has shown that there is not a single term-weighting model that would answer the best on any query.
Rather, per-query performance fluctuation among the term-weighting models has been shown.
This is called robustness problem of IR.
This dissertation has investigated the selective application of an appropriate term-weighting model on a per-query basis to alleviate the problem of robustness in retrieval effectiveness.
The objective of this dissertation is to characterize queries based on frequency distributions of their terms on document collections and try to predict which term-weighting model would be most effective for each query.
Thus, when an unseen query is submitted by the user, our approach will decide which term-weighting model should process it.
This section discusses the contributions and conclusions of this dissertation.

\subsection{Contributions}
The main contributions of this dissertation are the introduction of the STW framework and the proposed use of chi-square goodness-of-fit test  
on frequency distributions of query terms for identifying similar queries.
In addition, this dissertation draws insights from a large set of experiments, involving three different standard corpora, two different search tasks, two different document representations and two different effectiveness measures calculated at various cutoff levels.
This illustrates the generalizability of the STW framework.

Furthermore, we thoroughly evaluate the accuracy, effectiveness and robustness of the STW framework on two different retrieval tracks, namely Web Track and Million Query Track.
In particular, a Web collection that contains over a half billion English documents and about one thousand queries are used in this evaluation.

This study makes some important contributions to the body of existing work in both selective IR and robust IR. 
This dissertation presents experiments of the selective term-weighting for robust retrieval based on frequency distributions of query terms. 
This is the first examination of the frequency distributions of query terms on document collections in text-based IR. 
This has not been done before. 
As a by-product, a new family of query features can be driven from the frequency distribution of query terms for to use in IR research area.

This work presents a unique evaluation methodology for selective retrieval approaches when there exist multiple candidates to choose from.
Three aspects of such evaluation: accuracy, effectiveness and robustness are considered at the same time.
Two natural baselines that any selective retrieval approach should outperform at the minimum are derived and described.

The present dissertation also reveals the organic connection between the selective IR and the robust IR that focused on avoiding significant failures caused by the poorly-performing queries. 
This connection has much to do with the true understanding/definition of a significant failure, and an appreciation of it helps to gain insight into the selective retrieval approach.
Indeed, significant failure is a vague concept. 
When does a retrieval system fail significantly?
Can an effectiveness score of 0.2 or 0.6 be considered a failure for a particular query?

Whether a system performs poorly or not can only be meaningfully identified when it is \emph{relatively} compared to the \emph{other} systems.
For example, a system serves a query with the NDCG score of 0.6 and all the other systems attain NDCG score greater than 0.7. 
Since the model in question is the least effective, we can call NDCG score of 0.6 a significant failure.
On the other hand, a system can be the most effective with an NDCG score of 0.2 when the other systems return zero relevant documents.
Obviously NDCG score of 0.2 is not a significant failure in this case.

These examples clearly demonstrate that significant failure must be defined in a relative manner.
There must be other systems to compare with. 
The magnitude of an effectiveness score alone is not enough to define it. 
This is where selective retrieval approaches come into play.
The interesting relationship between the selective retrieval approaches and the problem of robustness in retrieval effectiveness is that the selective approaches are natural solutions to the robustness problem.


\subsection{Conclusions}
This section discusses the achievements and the conclusions of this study.
We tested our selective term weighting (STW) framework on the ClueWeb\{09\textbar12\} corpora and their corresponding TREC tasks.
In particular, we compared it with two random selection mechanisms as well as eight state-of-the-art term-weighting models, namely BM25, DFIC, DLH13, DPH, DFRee, PL2, LGD and the Language Modeling method with Dirichlet prior smoothing.
The experimental results showed that, our selective term-weighting approach does improve the average effectiveness compared with a baseline where a single term-weighting model is applied uniformly to all queries.

Our experimental results showed that the retrieval performance obtained by using our proposed STW framework could constantly outperform random selections and eight state-of-the-art models in the NDCG and MAP measures on different datasets. 
In addition, improvements were statistically significant in most cases.

Moreover, we investigated the robustness of our framework as measured by the GeoRisk.
Our experimental results showed that the STW framework is a truly robust system, which avoids significant per-query failures and also maintains a good average effectiveness at the same time.
We showed that more robust and more effective system can be built by leveraging existing term-weighting models in a selective manner, without inventing a new one.
The STW framework reached to a level of robustness that any single term-weighting cannot posses alone.

Furthermore, we showed that the queries that a system fails is as important as the queries that a system succeeds in a selective IR application.
We also empirically validated that the anchor text 100\% obeys the fundamental assumption of the frequentist approach to IR.
Finally, we compared our selective term-weighting method with a previous study by \citet{ModelSelection}, in which experimental results show that our approach performs better for the ClueWeb09 dataset. 

\section{Directions for Future Research}
This section discusses several directions for future work related to, or stemming from this dissertation.

\paragraph{Query features based on term frequency distributions}

The most common term statistics currently used in the IR literature are as follows: 
 
\begin{itemize}
  \item Document frequency: How many documents contain the term?
  \item Within-collection term frequency: How many times the term is observed in the entire collection?
  \item Number of documents: How many documents do exist in the entire collection?
  \item Number of terms: How many terms do exist in the entire collection?
\end{itemize}

These statistics has been leveraged in IDF and ICTF to quantify term specificity.
IDF is the logarithmically scaled inverse fraction of the documents that contain the word and the total number of documents.
ICTF simply uses ``the number of terms'' instead of the number of documents, thus ICTF is the ``\# terms'' counter part of the IDF.
Basically they both are based on the mean of the frequency distribution of the term in the collection.
However, there are three more statistics that can describe a frequency distribution of a term.
The four central moments used in mathematics and statistics are as follows: 
\begin{itemize}
  \item \textbf{Mean} is the first raw moment.
  \item \textbf{Variance} measures how far a set of numbers are spread out from their mean.
  \item \textbf{Skewness} is the measure of the lopsidedness of the distribution.
  \item \textbf{Kurtosis} is the measure of the heaviness of the tail of the distribution.
\end{itemize}

\citet*{Variance} argue that the term frequency distribution is pertinent to informativeness and the most informative terms tend to be those whose within-document frequency has high variance across a document collection.
The authors propose use of relative standard deviation as a measure of term specificity.

To the best of our knowledge, unlike the mean and variance \citep{Variance}, skewness and kurtosis have not ben utilized yet in the IR literature.
The computation of these features is more costly than IDF and ICTF, but they can obviously be included/used to describe terms in various IR tasks, such as query classification and learning to rank.
This new family of features based on term frequency distribution are waiting to be exploited by the IR community.

\bibliographystyle{elsarticle-harv}
\bibliography{References}
\end{document}